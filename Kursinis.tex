\documentclass[a4paper, 12pt]{article} %--< dokumento nustatymai >--%
%\usepackage[T1]{fontenc} %--< reikia kuomet norima naudoti komanda \k{a} - a raide bus parasyta su nosine >--%
\usepackage{graphicx} %--< reikalinga paveiksliuku ikelimui >--%
\usepackage[left=1.18in,top=0.79in,right=0.39in,bottom=0.79in]{geometry} %--< parasciu nustatymai >--%
\usepackage{color} %--< spalvoto srifto nustatymai >--%
\usepackage{xcolor}
\usepackage{setspace} %--< setspace.sty failas turi buti toje pacioje direktorijoje kaip ir winedt failas >--%
\usepackage{caption} %--< leidzia rasyti komentarus po objektais >--%
\usepackage{cite} %-< reikalinga citavimui >--%
\usepackage{verbatim} %--< leidzia naudoti \begin{comment}...\end{comment} funkcijas>--%

\usepackage[utf8]{inputenc} %--< kad butu galima rasyti lietuviskai >--%
\usepackage[L7x]{fontenc}
\usepackage[lithuanian]{babel}
\usepackage{lmodern}

%\usepackage[english]{babel}
\usepackage{array}
\usepackage[titles]{tocloft}
\usepackage{amsmath} %--< formuliu rasymui >--%
\usepackage{indentfirst} %--< pirmas paragrafas prasideda su tab'u >--%
\usepackage{enumitem} %--< tarpu tarp eiluciu sarase redagavimui >--%
%\usepackage{numline} %--< eiluciu numeravimas, sio paketo naudojimas panaikina section pavadinimus >--%
\usepackage{xfrac}
\usepackage{float}
\usepackage{array}
\usepackage{multirow}
\usepackage[]{SIunits}


\makeatletter
\let\c@lofdepth\relax
\let\c@lotdepth\relax
\makeatother
\usepackage{subfigure}
%\usepackage{biblatex}

\setlength{\cftaftertoctitleskip}{-10pt}
\setlength{\cftbeforetoctitleskip}{-10pt}
%\setlength{\parskip}{6pt} %--< kiek tekstas nutoles nuo paragrafo pavadinimo >--%
\parindent=36pt %--< tab'o dydis 1 inch = 72 pt >--%

%-----------< kaip atrodo paveiksleliu ir lenteliu tesktas >-----------------%
\DeclareCaptionLabelFormat{numbfirst}{#2\bothIfSecond{\nobreakspace}{pav}} % -< paveiksliukai vadinami "pav" >- %
\DeclareCaptionLabelFormat{numbtwo}{#2\bothIfSecond{\nobreakspace}{lentel\.{e}}} % -< lenteles vadinamos "lentele" >- %
\DeclareCaptionLabelSeparator{tarpas}{. } % -< po pav taskas ir tarpas >- %
\captionsetup{justification=centering, font=normalsize, labelfont={}, textfont={it}}



%-----------------------------< keletas naudingu komandu >------------------------------------------%
\begin{comment}

-----< Lietuviskos raides, jei neveikia rasymas lietuviskai >--------
\k{a} = ą
\v{z} = ž
\v{s} = š
\.{e} = ė 
\k{e} = ę
\={u} = ū
\k{u} = ų
\k{i} = į
\v{c} = č
----------------------------------------

\textbf{padaro teksta bold}
\textit{Padaro teksta italic}
%\cite{ka cituojam}
\newpage naujas puslapis

Jei norim tekste formuliu rasom ja tarp $formule$, jei numeruotai, tai pagal
\begin{equation}
formule
label{formules_label}
\end{equation}
ir naudojam formules citavimui \ref{formule_label}

\noindent pritaukia eilute prie krasto

-------< paveiksliukams patogu susikurti atskira folderi takim paveiksliukai ar pav >--------
-------< vienas paveiksliukas >---------
\begin{figure}[H] %--< nustatymai kur bus paveiksliukas h - kaip tekste, t - puslapio virsuje, b - puslapio apacioje, H - tiksliai kaip tekste, p - atskiram paveiksleliu pslapy >--%
\centering %--< lygiavimas >--%
\includegraphics[scale=0.44]{pav/SPP_dispersijos_kreive} %--< scale - dydis, {aplankas/pav.pavadinimas} >--%
\captionsetup{labelformat=numbfirst} %--< pries tai apibrezem kaip turi atrodyti tekstas po paveiksleliu >--%
 \captionsetup{labelseparator=tarpas}
 \caption{Tesktas po paveiksleliu}
 \label{vienas}
 \end{figure}

-------< du paveiksliukai ir t.t. >---------
\begin{figure}[h]
\centering
\subfigure[]{ %--< pirmas paveiksliukas, jei [] tusti tai numeruojama raidemis >--%
\includegraphics[scale=0.3]{pav/pirmas}
\label{pirmas}
}
\centering
\subfigure[]{ %--< antras paveiksliukas >--%
\includegraphics[scale=0.27]{antras}
 \label{antras}
 }
\captionsetup{labelformat=numbfirst}
\captionsetup{labelseparator=tarpas}
\caption{Kas paveikslikuose (a) ir (b), tesktas po paveiksliuku}
\label{dupaveiksliukai}
\end{figure}

----------< numeruojam teksta >-------
\begin{enumerate}[itemsep=12pt, parsep=-10pt]
 \item[a)]  %--< jei [] nieko arba be [], tai numeruos skaiciais
\item[b)] 
\item[c)]      
\end{enumerate}

\end{comment}

%=================================================================%
%--< dalis kuria reikia uzpildyti, bet tam tikrais atvejais gali tekti pagal poreikius paredaguoti title.tex faila >--%
%=================================================================%

\newcommand{\studentas}{Mindaugas Kurmauskas} %--< cia irasyti savo varda pavarde (toliau kitus duomenis) ir nebereikes tekste >--%
\newcommand{\vadovas}{dr. Mindaugas Vil\={u}nas}
\newcommand{\recenzentas}{dr. Gytis Sliau\v{z}ys}
\newcommand{\vedejas}{dr.(HP) K\k{e}stutis Arlauskas}
\newcommand{\pavadinimas}{Cortex R4 architekt\={u}ros mikrovaldiklio savybi\k{u} tyrimas} %--< cia irasyti darbo pavadinima >--%

\newcommand{\katedra}{\kkek} %--< irasyti savo, nebutinai komanda >--%
%=======%
\newcommand{\kkek}{Kieto k\={u}no elektronikos katedra}
\newcommand{\kek}{Kvantin\.{e}s elektronikos katedra}
\newcommand{\pfk}{Puslaidininki\k{u} fizikos katedra}
\newcommand{\rfk}{Radiofizikos katedra}
\newcommand{\tfk}{Teorin\.{e}s fizikos katedra}
\newcommand{\bfsk}{Bendrosios fizikos ir spektroskopijos katedra}
%==========%

\newcommand{\darbas}{\bakis} %--< irasyti savo >--%
%=========%
\newcommand{\bakikurs}{Pagrindini\k{u} studij\k{u} kursinis darbas}
\newcommand{\bakis}{Pagrindini\k{u} studij\k{u} baigiamasis darbas}
\newcommand{\magikurs}{Magistrant\={u}ros studij\k{u} kursinis darbas}
\newcommand{\magis}{Magistrant\={u}ros studij\k{u} baigiamasis darbas}
%==========%

\newcommand{\studprog}{\tf} %--< irasyti savo, nebutinai komanda >--%
%==========%
%--< besimokantiems fizfake studiju programos >--%
\newcommand{\fiz}{Fizika}
\newcommand{\tf}{Taikomoji fizika}
\newcommand{\tfe}{Telekomunikacij\={u} fizika ir elektronika}
\newcommand{\kf}{Kompiuterin\.{e} fizika}
\newcommand{\mtfv}{Moderni\k{u}j\k{u} technologij\k{u} fizika ir vadyba}
\newcommand{\bef}{Branduolin\.{e}s energetikos fizika}
\newcommand{\apchef}{Aplinkos ir chemin\.{e} fizika}
\newcommand{\bio}{Biofizika}
\newcommand{\laztech}{Lazerin\.{e}s technologijos}
\newcommand{\fiztech}{Fizikin\.{e}s technologijos ir j\k{u} vadyba}
\newcommand{\lazfiz}{Lazerin\.{e} fizika ir optin\.{e}s technologijos}
\newcommand{\mpf}{Med\v{z}iagotyra ir puslaidininki\k{u} fizika}
\newcommand{\optel}{Optoelektronikos med\v{z}iagos ir technologijos}
\newcommand{\tfem}{Telekomunikacij\k{u} fizika ir elektronika}
\newcommand{\tfa}{Teorin\.{e} fizika ir astrofizika}
%=================================================================%
%=================================================================%


%--------------------------------------< nuo cia prasideda dokumento tekstas >-------------------------------------%


\begin{document} % visas tekstas turi buti tarp begin{document} ir end{document} komandu


\begin{titlepage}

\begin{center}


% Upper part of the page

\MakeUppercase{\normalsize \textbf{Vilniaus universitetas}}\\
\MakeUppercase{\normalsize \textbf{Fizikos fakultetas}}\\
\MakeUppercase{\normalsize \textbf{\katedra}}\\[180pt]
\normalsize \studentas \\ [24pt]

% Title

\MakeUppercase{{ \normalsize \pavadinimas}}\\[24pt]

\normalsize \darbas \\ [24pt]
(studij\k{u} programa -- \MakeUppercase{\normalsize \studprog)}\\ [130pt]

\linespread{1.3}
% Author and supervisor
\begin{minipage}{0.7\textwidth}
\begin{flushleft} \normalsize

Studentas \\  %--< nekeiciam >--%
Darbo vadovas \\ %--< nekeiciam >--%
Recenzentas\\ %--< nekeiciam >--%
Katedros ved\.{e}jas %--< nekeiciam >--%
\end{flushleft}
\end{minipage}
\begin{minipage}{0.29\textwidth}
\begin{flushleft} \normalsize
\studentas \\
\vadovas \\
\recenzentas \\
\vedejas
\end{flushleft}
\end{minipage}

\vfill

% Bottom of the page
{\normalsize Vilnius \the\year}

\end{center}

\end{titlepage}  % itraukiam titulini lapa i visa dokumenta

\newpage

\pagenumbering{arabic}
\setcounter{page}{2} % nuo kurio skaitmens pradeda numeruoti puslapius

%\tocloftpagestyle{abbrv}
\renewcommand{\cftsecleader}{\cftdotfill{\cftdotsep}} % turinio isvaizdos koregavimas

\renewcommand{\multirowsetup}{\centering}

\renewcommand{\contentsname}{Turinys}
\renewcommand\refname{Literat\={u}ros s\k{a}ra\v{s}as} % pakeicia "references" i "literaturos sarasas"
\tableofcontents

\newpage

\begin{onehalfspacing}

\section*{\k{I}vadas} % * nurodo nenumeruoti sitos dalies

\addcontentsline{toc}{section}{\k{I}vadas} % prideda prie turinio

Egzistuoja daug u\v{z}duo\v{c}i\k{u} kurioms atlikti reikia didelio patikimumo. J\k{u} atlikimui galima naudoti standartinius mikrovaldiklius, kombinuojant su programin\.emiais sprendimais. Tokiais kaip WDT $($watchdog timer$)$, kuris pri\v{z}i\=uri pagrindin\k{e} program\k{a}, kad ji tinkamai atlikt\k{u} savo darb\k{a}. Nuolatiniais testais patikrinti duomen\k{u} skai\v{c}iavimo patikimum\k{a}, pavyzd\v{z}iui atlikti algoritm\k{a} du kartus ir tikrinti ar atsakymai sutampa. Daugeliu atveju visos papildomos priemon\.es u\v{z}tikrinti patikimum\k{a} prailgina duomen\k{u} skai\v{c}iavimo laik\k{a}. \\
\indent Sistemose kurioms reikalingas didelis patikimumas ir darbas realeame laike labai praver\v{c}ia sistemos dubliavimas su tikrinimo funkcijomis. Klaidos tikimyb\.e ne visais atvejais yra tolydi laike, taigi klaidos gali tuo pa\v{c}iu metu ivykti abiejose sistemose. Tokiu atveju naudinga, kai sistemos dirba pastumtos laike at\v{z}vilgiu viena kitos. \\
\indent Yra daug aplink\k{u} kur gali reik\.eti toki\k{u} sistem\k{u}. Pramonin\.es saugumo reikalaujan\v{c}ios aplinkos, tokios kaip:
\begin{itemize}
\item automobil\k{u} stabdymo sistemos, tokios kaip rat\k{u} anti blokavimo sistema $($ABS$)$
\item elektros gaminimo ir paskirstymo sistemos.  
\item liftai ir eskalatoriai
\end{itemize}      
Medicinoje:
\begin{itemize}
\item defibriliatoriai
\item radiacijos terapija
\item robotizuotos operacijos
\end{itemize}
\noindent 
\indent Atsi\v{z}velgiant \k{i} \v{s}\k{i} poreik\k{i} kompanija
 Teksas Instruments neseniai i\v{s}leido ARM Cortex-R4 mikrovaldikli\k{u} \v{s}eim\k{a} 
 kodiniu pavadinimu Hercules. Mums kilo klausimas koks yra \v{s}i\k{u}
  mikrovaldikli\k{u} energetinis efektyvumas, bei mikrovaldikli\k{u} veikimas esant ekstramaliam poveikiui, k\k{a} \v{s}iame
   darba ir bandoma i\v{s}siai\v{s}kinti. \\
%\indent Tolimesniame darbe b\=ut\k{u} idomu patikrinti veikim\k{a} esant ekstremaliam poveikui, galimyb\k{e} panaudoti vien\k{a} i\v{s} Hercules \v{s}eimos atstov\k{u} RM48 kosmose, kur jo u\v{z}duotis gal\.et\k{u} b\=uti navigacija, kuriai atlikti reikia slankaus formato ir skai\v{c}i\k{u} tikslumo.  %

\newpage

\section{ARM architekt\={u}ra}
ARM $($advanced RISC mashines$)$ Holdings ank\v{s}\v{c}iau buvusi Acorn Computers. Acorn Computers prad\.{e}jo savo versl\k{a} kaip mikrovaldikli\k{u} gamintojas. ARM Holdings pati nebamina mikrovaldikli\k{u}, bet pardavin\.{e}ja architekt\={u}ros dizainus puslaidininkini\k{u} \k{i}tais\k{u} gamintojams. ARM licenzijuoti 32 bit\k{u} architekt\={u}ros mikrovaldikliai yra pirmaujantys pasaulyje pagal pagaminimimo kiek\k{i}. \v{S}iai s\.{e}kmei didel\k{e} \k{i}tak\k{a} turi geras grei\v{c}io ir vartojamos galios santykis, kas leid\v{z}ia juos patogiai naudoti ne\v{s}iojamuose \k{i}renginiuose.    \\
\indent ARM architekt\={u}ra paremta RISC\footnote{RISC - reduced instruction set computing.} - paprastesni\k{u} komand\k{u} sistema:
\begin{itemize} 
\item daug bendrosios paskirties registr\k{u}
\item paprastai tiesiogiai dirbama su registrais, o ne su atmintimi
\item paprasti adresavimo re\v{z}imai
\item vienodo fiksuoto ilgio instrukcijos
\end{itemize} 
Tai pat ARM architekt\={u}ra papildomai turi:
\begin{itemize}
\item tiesiogin\k{e} ALU\footnote{ALU - arithmetic logic unit} - aritmetinio loginio ir post\={u}mio \k{i}rengini\k{u} kontrol\k{e}
\item automati\v{s}kai did\.{e}jan\v{c}ius ir automati\v{s}kai ma\v{z}\.{e}jan\v{c}ius adresavimo re\v{z}imus cikl\k{u} optimizavimui
\item i\v{s} karto keli\k{u} instukcij\k{u} \k{i}k\.elimo ir pa\.{e}mimo komandas duomen\k{u} pralaidumo optimizavimui
\item beveik vis\k{u} komand\k{u} s\k{a}lygin\k{i} vykdim\k{a} vykdymo pralaidumui padidinti
\item Thumb instrukcij\k{u} prapl\.{e}tim\k{a}, kuris leid\v{z}ia vykdyti 16bit\k{u} komandas vietoje 32bit\k{u}. Tai leid\v{z}ia padidinti kodo tank{i} neilginant komand\k{u} atlikimo trukm\.{e}s
\end{itemize}
Paprastai kiekviena instrukcija atliekama per vien\k{a} takt\k{a}, tad yra nesunku nusp\.{e}ti programos vykdym\k{a} ir galima nesunkiai pritaikyti kaskadin\k{i} komand\k{u} vykdym\k{a}. Komand\k{u} atkodavimas reikalauja ma\v{z}iau tranzistori\k{u}, nei mikrovaldikliams paremtiems sud\.{e}tingesni\k{u} komand\k{u} sistema $($pvz. CISC\footnote{CISC - complex instruction set computing.}$)$ - kur viena komanda yra keli\k{u} instrukcij\k{u} rinkinys. Dabartiniai ARM mikrovaldikliai yra 32bit\k{u} instrukcijos ilgio, bei 32bit\k{u} adresavimo. Yra paruo\v{s}tos ir 64bit\k{u} virsijos. \\
\indent ARM turi 31 bendrosios paskirties 32bit\k{u} ilgio registrus. Betkuriuo metu galima matyti 16 j\k{u}. Kiti registrai yra naudojami i\v{s}im\v{c}i\k{u} apdorojimui pagreitinti. Visos intstrukcijos skirtos registrams gali naudoti betkur\k{i} i\v{s} 16 registr\k{u}. I\v{s} 16 visad matom\k{u} registru keli turi specializuotas funkcijas yprastas visoms architekt\k{u}roms:
\begin{itemize}
\item Rietuv\.{e}s rodykl\.{e} $($ang. stack pointer$)$ - paprastai 13 registras.
\item Nuorodos registras $($ang. link register$)$ - 14 registras naudojamas kaip adresas \k{i} kur\k{i} einama funkcijos i\v{s}kvietimui pasibaigus. 14 registras gali b\={u}ti naudojamas kaip bendrosios paskirties registras.
\item Programos skaitiklis $($ang. program counter$)$ - 15 registras. Naudojamas kaip nuoroda \k{i} adres\k{a} sekan\v{c}ios instrukcijos. 
\end{itemize}
Lik\k{e} 13 registr\k{u} speciali\k{u} funkcij\k{u} neatlieka. \\
\indent ARM7 ir naujesn\.{e}s architekt\={u}ros palaiko 7 r\={u}\v{s}i\k{u} i\v{s}imtis ir privilegijuot\k{a} apdorojimo re\v{z}im\k{a} kiekvienai i\v{s} j\k{u}:
\begin{itemize}
\item reset
\item bandymas atlikti ne\v{z}inomos instrukcijos
\item programos pertraukties instrukcijos $($SWI - software interupt instruction$)$. Gali b\={u}ti naudojamos i\v{s}kviesti operacin\k{e} sistem\k{a}
\item i\v{s}ankstinio duomen\k{u} pa\.{e}mimo at\v{s}aukimas $($ang. prefetch abort$)$
\item normalus tr\={u}kis $($IRQ - interrupt request$)$
\item greitas tr\={u}kis $($FIQ - fast interrupt request$)$
\end{itemize}
\k{I}vykus i\v{s}im\v{c}iai kaikurie standartiniai registrai yra pakei\v{c}iami registrais specifiniais tai i\v{s}im\v{c}iai. 14-tame registre laikomas adresas \k{i} kur\k{i} gri\v{z}tama po i\v{s}imties \k{i}vykdymo. 13 registras yra i\v{s}saugomas, kad i\v{s}imtis gal\.{e}t\k{u} naudotis savo rietuve, netrigdydama tolimesnio programos darbo. Greito tr\k{u}kio re\v{z}ime yra i\v{s}augomi registrai nuo 8 iki 12, kad nereik\.{e}t\k{u} j\k{u} saugoti atskirai ir b\={u}t\k{u} galima jais greitai naudotis. \\
\indent \k{I}vykus i\v{s}im\v{c}iai \v{s}erdis sustabdo programos vykdyma apibr\.{e}\v{z}tu b\={u}du ir pereina prie i\v{s}imties vykdymo \k{i} viena i\v{s} fiksuot\k{u} adres\k{u} atmintyje, \v{z}inom\k{u} kaip i\v{s}imties vektoriai $($ang. exception vector$)$. Yra atskiras vektorius kiekvienai i\v{s}im\v{c}iai, \k{i}skaitant ir reset. Veikimas yra apibr\.{e}\v{z}tas normaliam programos darbui ir klaid\k{u} radimo re\v{z}imui $($ang. debug mode$)$.  \\
\indent Visa branduolio b\={u}kl\.{e} yra laikoma statuso registe vadinamu dabartin\.{e}s pragramos statuso registru $($CPSR - current program status register$)$, kuriame yra:
\begin{itemize}
\item keturios aritmetin\.{e}s v\.{e}lev\.{e}l\.{e}s $($neigiamas, nulis, perkelti ir perpildymo$)$
\item 2 bitai tr\={u}ki\k{u} i\v{s}jungimui
\item 5 bitai procesoriaus darbo re\v{z}imui
\item 2 bitai nurodantys kuris instrukcij\k{u} rinkinys naudojamas: ARM, Thumb ar Jazelle  $($ naujesniuose procesoriuose palaikomas JAVA instrukcij\k{u} rinkinys$)$
\end{itemize}
\subsection{ARM Cortex-A architekt\={u}ra}
Mikrovaldikliai skirti dirbti su operacin\.{e}mis sistemomis tokiomis kaip Linux, Android, Microsoft Windows. \v{S}ios klas\.{e}s mikrovaldikliai turi atminties valdymo \k{i}taisa, kuris padeda valdyti atmint\k{i} operacin\.{e}ms sistemoms ir \k{i}galina parsiuntim\k{a} ir paleidim\k{a} kit\k{u} program\k{u}. Galimi panaudojimai:
\begin{itemize}
\item i\v{s}manieji telefonai
\item plan\v{s}etiniai kompiuteriai
\item skaitmenin\.{e}je televizijoje
\item navigacin\.{e}je sistemoje
\item protinguose spauzdintuvuose
\end{itemize}

\subsection{ARM Cortex-R architekt\={u}ra}
ARM integruoti realaus laiko apdorojimo mikrovaldikliai. Turi didel\k{i} taktin\k{i} da\v{z}n\k{i}, ilg\k{a} kaskadin\k{e} linij\k{a}. Da\v{z}niausiai \v{s}ie mikrovaldikliai naudoja realaus laiko operacin\k{e} sistem\k{a} $($RTOS - real-time operating system$)$ kartu su vartotojo sukurta programa, kuriai u\v{z}tenka atminties apsaugos \k{i}renginio. Taikymas:
\begin{itemize}
\item automobili\k{u} kontrol\.{e}s sistemose
\item bevieli\k{u} ir laidini\k{u} sensori\k{u} tinkluose
\item spauzdintuvuose
\item tinklo \k{i}rangoje
\end{itemize}
\subsubsection{Cortex-R4}
Paremtas ARMv7-R architekt\={u}ra. 8 pakop\k{u} kaskadin\.{e} linija su i\v{s}ankstiniu duomen\k{u} nuskaitymu ir \v{s}akojimosi nusp\.{e}jimu. Palaiko ARM ir Thumb-2 instrukcijas. Harvardo instrukcij\k{u} ir duomen\k{u} atskyrimas. Atminties apsaugos \k{i}renginys. Dvi \v{s}erdys dubliuojan\v{c}ios skai\v{c}iavimus.  
\subsubsection{Cortex-R5}
8 pakop\k{u} kaskadin\.{e} linija. Palaiko ARM ir Thumb-2 instrukcijas. Harvardo instrukcij\k{u} ir duomen\k{u} atskyrimas. Slankaus kablelio \k{i}renginys. Ma\v{z}o delsimo periferijos prievadai. Prapl\.{e}sta atminties apsauga apsaugo ir periferij\k{a}. Dvi \v{s}erdys veikian\v{c}ios kartu arba atskirai.
\subsubsection{Cortex-R7}
Pa\v{z}angiausias Cortex-R \v{s}eimoje. 11 pakop\k{u} kaskadin\.{e} linija su nei\v{s}eil\.{e}s instrukcij\k{u} vykdimo galimybe. Dvi \v{s}erdys gali veikti dubliuoto skai\v{c}iavimo, simetrinio ir asimetrinio darbo re\v{z}imais. Pasirinktinas slankaus kablelio \k{i}renginys.  

\subsection{ARM Cortex-M architekt\={u}ra}
Mikrovaldikliai skirti integruotiems \k{i}rengiriams. Pasi\v{z}ymi ma\v{z}omis enegrijos s\k{a}naudomis. Architekt\={u}rinis miego r\.{e}\v{z}im\k{u} palaikymas. \\
Mikrovaldikliai:
\begin{itemize}
\item ARM Cortex-M0:
\begin{itemize}
\item 8/16 bit\k{u} programoms
\item pigus ir paprastas
\end{itemize}
\item ARM Cortex-M0+
\begin{itemize}
\item 8/16 bit\k{u} programoms
\item pigus ir energeti\v{s}kai efektyviausias
\end{itemize}
\item ARM Cortex-M3
\begin{itemize}
\item 16/32 bit\k{u} programoms
\item greitas bendros paskirties mikrovaldiklis
\end{itemize}
\item ARM Cortex-M4
\begin{itemize}
\item 32 bit\k{u} programoms
\item efektyvus skaitmenini\k{u} signal\k{u} valdiklis
\end{itemize}

\end{itemize}

%--<
%\subsection{ARM mikrovaldikli\k{u} architekt\={u}ros \v{s}eimos}
%ARM sukurt\k{u} mikrovaldikli\k{u} \v{s}eimos, architekt\k{u}ros ir trumpi apra\v{s}ai:
%\begin{itemize}
%\item ARM1 $($ARMv1 architekt\={u}ra$)$ - pirmasis kompanijos mikrovaldiklis.
%\item ARM2 $($ARMv2 ir ARMv2a$)$ - patobulintas ARM1. 8MHz taktinis da\v{z}nis.
%\item ARM3 $($ARMv2a$)$ - pirmieji mikrovaldikliai su integruota atminties talpykla. 4KB atminties, 25MHz taktinis da\v{z}nis
%\item ARM6 $($ARMv3$)$
%\begin{itemize}
%\item ARM60 - pirmasis palaik\.{e} 32bit\k{u} atminties adresavimo erdv\k{e} $($ank\v{s}\v{c}iau b\={u}davo 26bit\k{u}$)$. 25MHz
%\item ARM600 - kaip ARM60 tik su integruota atmintimi ir slankaus kablelio moduliu. 4KB atminties ir 12MHz taktinis da\v{z}nis.
%\item ARM610 - kaip ARM60 tik su integruota atmintimi. 4KB atminties ir 33MHz taktinis da\v{z}nis.
%\end{itemize}
%\item ARM7 $($ARMv3$)$
%\item ARM7TDMI $($ARMv4T$)$- 3 pakop\k{u} kaskadinis instrukcij\k{u} vykdymas. Thumb instrukcijos
%\item ARM7EJ $($ARMv5EJ$)$ - 5 pakop\k{u} kaskadinis instrukcij\k{u} vykdymas. Thumb, Jazelle ir auk\v{s}tensnio lygio signal\k{u} apdorojimo instrukcijos.
%\item ARM8 $($ARMv4$)$ - 5 pakop\k{u} kaskadinis instrukcij\k{u} vykdymas su statiniu \v{s}akojimosi nusp\.{e}jimu. 72MHz
%\item ARM9TDMI $($ARMv4T$)$ - 5 pakop\k{u} kaskadinis instrukcij\k{u} vykdymas, Thumb instrukcijos, 16KB integruota atmintis. 180MHz
%\item 

%\end{itemize}

\section{Naudojamos \k{i}rangos ir programos}
Pasirinktas Cortex-R4 mikrovaldiklis d\.{e}l dviej\k{u} \v{s}erdzi\k{u} dubliuoto skai\v{c}iavimo ir paprastumo lyginant su kitais Cortex-R klas\.{e}s mikrovaldikliais. Naudojamas Texas Instruments TMDXRM48USB maketas. Jame yra:
\begin{itemize}
\item Mikrovaldiklis xRM48L950AZWTT:
\begin{itemize}
\item Du 32-bitu ARM Cortex-R4F procesoriai, veikiantys kartu\footnote{lockstep - kartu atlieka tas pa\v{c}ias komandas}
\item 3MB flash, 256kB RAM
\item \v{S}erdies taktinis da\v{z}nis 200MHz
\item fiksuoto tikslumo (32bitu) ir dvigubo tikslumo (64bitu) slankaus kablelio aritmetikos modulis
\end{itemize} 
\item Integruotas XDS100v2 emuliatorius programavimui per usb
\item LED'ai, temperat\=uros sensorius, \v{s}viesos sensorius, akselerometras
\end{itemize}
%%%%%%%%%%%%%%%%%%%%%%%%%%%%%%%%%%%%%%%%%%%%%%%%%%%%%%%%%%%%%%%%%%%%%%%%%%%
Palyginimui pasirinktas Cortex-M4 architekt\={u}ros mikrovaldiklis, d\.{e}l savo skai\v{c}iavim\k{u} galingumo pana\v{s}umo ir energetinio efektyvumo. Pasirinktas STM32F4DISCOVERY maketas, d\.{e}l laiko patikrintos technologijos.  
\begin{itemize}
\item mikrovaldiklis STM32F407VGT:
\begin{itemize}
\item 32-bitu ARM Cortex-M4F procesorius
\item 1MB flash, 192 kB RAM 
\item taktinis da\v{z}nis iki 168MHz 
\item fiksuoto tikslumo (32bitu) ir slankaus (32bitu) kablelio aritmetikos modulis
\end{itemize}
\item Integruotas ST-LINK/V2 emuliatorius programavimui per usb
\item LED'ai, akselerometras, skaitmeninis mikrofonas
\item CS43L22- SAK garsui su integruotu D klas\.es garso stiprinimu
\end{itemize}
Pasirinktas kompiliatorius - IAR 6.4. Darbo prad\.ejimo metu vienintelis palaik\.e abu maketus be papildomos \k{i}rangos. 
\subsection{xRM48L950 apra\v{s}as}
 
\subsubsection{Procesorius}
xRM48L950 turi dvi integruotas 32-bit\k{u} RISC ARM Cortex-R4F \v{s}erdis, su slankaus kablelio aritmetikos moduliu. \v{S}erdys i\v{s}destyti skirtinga orientacija  
\begin{figure}[H] %--< nustatymai kur bus paveiksliukas h - kaip tekste, t - puslapio virsuje, b - puslapio apacioje, H - tiksliai kaip tekste, p - atskiram paveiksleliu pslapy >--%
\centering %--< lygiavimas >--%
\includegraphics[scale=2.5]{pav/orentacija.jpg} %--< scale - dydis, {aplankas/pav.pavadinimas} >--%
\captionsetup{labelformat=numbfirst} %--< pries tai apibrezem kaip turi atrodyti tekstas po paveiksleliu >--%
\captionsetup{labelseparator=tarpas}
\caption{\v{S}erdys pakreipitos fizi\v{s}kai viena kitos at\v{z}vilgiu korpuse}
\label{vienas}
\end{figure}

Taip pat taktinis da\v{z}nis procesoriams paduodamas su 2 takt\k{u} u\v{z}laikymo skirtumu. Procesoriai fizi\v{s}kai atskirti 100$\micro$m. atstumu.  
Procesori\k{u} i\v{s}\.ejimo signalai palyginami atskirame modulyje.
\begin{figure}[H] %--< nustatymai kur bus paveiksliukas h - kaip tekste, t - puslapio virsuje, b - puslapio apacioje, H - tiksliai kaip tekste, p - atskiram paveiksleliu pslapy >--%
\centering %--< lygiavimas >--%
\includegraphics[scale=2.8]{pav/impl.jpg} %--< scale - dydis, {aplankas/pav.pavadinimas} >--%
\captionsetup{labelformat=numbfirst} %--< pries tai apibrezem kaip turi atrodyti tekstas po paveiksleliu >--%
\captionsetup{labelseparator=tarpas}
\caption{Dviej\k{u} branduoli\k{u} \k{i}gyvendinimas}
\label{vienas}
\end{figure}
\k{I}renginys palaiko sav\k{e}s testavim\k{a}. Jo metu galima priversti branduoli\k{u} i\v{s}\.ejimo signal\k{u} nesutapim\k{a}, taip patikrinama ar veikia branduoli\k{u} palyginimo modulis ir ar \k{i}renginys sugeba atpa\v{z}inti klaidas. \v{S}erdys maitinamos 1.2V \k{i}tampa. 


\subsubsection{Atmintys}
\k{I}renginys palaiko little-endian $($LE32$)$ format\k{a}. Tai rei\v{s}kia, kad atmintyje jauniausias bitas yra saugomas pirmas.
SRAM ir Flash atminys turi ECC\footnote{ECC - error corection code} apsaug\k{a} - galimyb\.e 1 bito klaidos aptikimui ir pataisymui, bei 2 bit\k{u} klaidos aptikimui 8bit\k{u} bloke. Flash atmintis maitinama 3.3V \k{i}tampa. Esant kaskadiniui
re\v{z}imui flash atmintis gali veikti iki 200MHz taktiniu da\v{z}niu. SRAM atmint\k{i} galima nuskaityti arba ira\v{s}yti 1 ciklu nepriklausomai nuo re\v{z}imo.   


\subsubsection{Tr\=uki\k{u} sistema}
Vektorinis tr\=uki\k{u} valdiklis \k{i}galina veiksm\k{u} prioritizavim\k{a} ir kontrol\k{e}. Tr\=ukis - tai paprogram\.es i\v{s}kvietimas, galimai nutraukiant esam\k{a} procesoriaus veikl\k{a}. Paprastai toks \k{i}vykis reikalauja greito procesoriaus atsako. Procesorius i\v{s} normalios programos per\v{s}oka \k{i} tr\=ukio aptarnavimo paprogram\k{e}. RM48 palaiko 96 skirtingus programi\v{s}kai reguliuojamo prioriteto tr\=ukius. Yra 2 tr\=ukio vektoriai - normal\=us tr\=ukiai $($IRQ\footnote{IRQ - interrupt request}$)$ ir greitieji tr\=ukiai $($FIQ\footnote{FIQ - fast interrupt request}$)$. Greitieji tr\=ukiai turi didesn\k{i} prioriteta u\v{z} normalius ir yra nemaskuojami t.y. juos \k{i}jungus j\k{u} neimanoma i\v{s}jungti ar pakeisti, nebent atlikus sistemos perkrovima, kai visi tr\=ukiai yra automati\v{s}kai i\v{s}jungiami. Greitieji tr\=ukiai gali pertraukti normalius.  


\subsubsection{Prievadai}
RM48L950 turi 2 laikma\v{c}i\k{u} $($N2HET\footnote{N2HET - next generation high end timer}$)$ koprocesorius su tiesiogine magistrales prieiga $($DMA$)$ realaus laiko kontrolei. J\k{u} pagalba galima atlikti \v{i}tampos impulso plo\v{c}io moduliacij\k{a} $($PWM$)$, steb\.eti i\v{s}vado login\k{i} lyg\k{i}, bei gali veikti kaip \k{i}vesties/i\v{s}vesties periferija. Taipat yra du integruoti 12 bit\k{u} rezoliucijos analogas kodas keitikliai. Prievadai maitinami 3.3V \k{i}tampa. Taipat yra bendrosios paskirties \k{i}vesties i\v{v}esties prievadai.
Palaiko \v{s}ias periferijas:
\begin{itemize}
\item USB
\item Ethernet
\item UART
\item MibSPI
\item $I^2C$
\item SCI
\item DCAN
\end{itemize}



\subsection{STM32F4 apra\v{s}as}


\subsubsection{Procesorius}
ARM Cortex-M4F 32-bit\k{u} RISC procesorius, su slankaus kablelio aritmetikos moduliu. Palaiko DSP instrukcijas ir 32bit\k{u} slankaus kablelio duomenis.      


\subsubsection{Atmintys}
Adaptyvus realaus laiko atminties greitintuvas leid\v{z}ia minimal\k{u} procesoriaus laukim\k{a} esant dideliems taktavimo da\v{z}niams nuskaitant ir \k{i}ra\v{s}ant \k{i} Flash atmint\k{i}. SRAM atmintis nuskaitom ir \k{i}ra\v{s}oma per 1 takt\k{a}. Atminties apsaugos modulis riboja procesoriaus pri\.ejim\k{a} prie atminties, kad b\=ut\k{u} galima i\v{s}vengti nety\v{c}ini\k{u} atminties sugadinim\k{u}. Tai ypa\v{c} praver\v{c}ia kai turima kriti\v{s}kai svarbi\k{u} duomen\k{u} ir juos reikia apsaugoti nuo kit\k{u} veikian\v{c}i\k{u} proces\k{u}. Pavyzd\v{z}iui, jeigu yra veikianti operacin\.e sistema procesoriuje.


\subsubsection{Tr\=uki\k{u} sistema}
Palaiko \k{i}d\.etin\k{e} vektorin\k{e} tr\=uki\k{u} kontrol\k{e} $($NVIC\footnote{NVIC - nested vectored interrupt controller}$)$.
Yra 16 prioritet\k{u} lygi\k{u}, 82 maskuojami tr\=uki\k{u} kanalai. Leid\v{z}ia auk\v{s}tesnio prioriteto tr\=ukiams pertraukti vykdomus \v{z}emesnio prioriteto tr\=ukius, minimaliai apkraunant procesori\k{u}. 


\subsubsection{Prievadai}    
140 \k{i}\.ejimo i\v{s}\.ejimo bendrosios paskirties prievad\k{u}, su tr\=ukio generavimo galimybe. Periferijos greitis - iki 84MHz. Du 12 bit\k{u} skaitmeninis analogas keitikliai. Trys 12 bit\k{u} analogas kodas keitikliai. 
Palaikomos periferijos:
\begin{itemize}
\item $I^2C$
\item SPI
\item UART
\item $I^2S$
\item CAN
\item USB 2.0
\item Ethernet
\end{itemize}

\newpage %perkelia testa i nauja lapa
%\section{Ka kas ir kaip}
%%%%%%%%%%%%%%%%%%%%%%%%%%%%%%%%%%%%%%%%%%%%%%%%%%%%
%
%%%%%%%%%%%%%%%%%%%%%%%%%%%%%%%%%%%%%%%%%%%%%%%%%%%%
\section{Naudoti testavimo algoritmai ir j\k{u} apra\v{s}ai}
%\addcontentsline{toc}{section}{Praktin\.{e}\ dalis} % prideda
\subsection{Matematini\k{u} funkcij\k{u} skai\v{c}iavimo grei\v{c}i\k{u} \k{i}vertinimo algoritmai}
\v{S}iame paragrafe apra\v{s}ytiems algoritmams buvo naudojamas laiko tr\=ukis, kurio pagalba buvo ska\v{c}iuojamas algoritmo atlikimo greitis 1ms tikslumu. Abu maketai buvo nustatyti veikti 100MHz grei\v{c}iu. \v{S}iais matavimais siekta patikrinti Cortex-R4 64bit\k{u} slankaus kablelio \k{i}taiso \k{i}tak\k{a} skai\v{c}iavimo grei\v{c}iui. Algoritmai buvo leisti 100 kart\k{u} ir fiksuoti u\v{z}duoties atlikimo laikai suvidurkinti. Tik\.{e}tasi, kad Cortex-R4 mikrovaldiklis dvigubo tikslumo slankaus kablelio kintam\k{u}j\k{u} algoritmus atliks grei\v{c}iau nei Cortex-M4 mikrovaldiklis, bet kit\k{u} skai\v{c}iavum\k{u} atlikimo laikai b\={u}t\k{u} pana\v{s}\={u}s.
\subsubsection{Slankaus kablelio algoritmai}
Slankaus kablelio Gauss Legendre algoritmas $\pi$ skai\v{c}iavimui$[$1$]$:
\begin{equation*}\label{xx}
a_{0}=1\;b_{0}=\frac{1}{\sqrt{2}}\;t_{0}=\frac{1}{4}\;p_{0}=1\\
\end{equation*}
\begin{equation*}
a_{n+1} = \frac{a_n + b_n}{2},
\end{equation*}
\begin{equation*}
b_{n+1} = \sqrt{a_n b_n}, 
\end{equation*}
\begin{equation*}
t_{n+1} = t_n - p_n(a_n - a_{n+1})^2, 
\end{equation*}
\begin{equation*}
p_{n+1} = 2p_n. 
\end{equation*}
\begin{equation}
\pi \approx \frac{(a_n+b_n)^2}{4t_n}
\end{equation}
Naudojant dvigubo tikslumo kintam\k{u}j\k{u} test\k{a} buvo ie\v{s}komas 1000 narys, suskai\v{c}iuota  
konstanta nuo pasirinktosios skiriasi $3.55271 * 10^{-15}$. 
%\textcolor{gray}{\\
\begin{verbatim}
  int doubleTest() { 
    volatile int i; 
    volatile double an,bn,tn,pi; 
    volatile double a,b,t,p;  
    a = 1.0; b = 1/sqrt(2); 
    t = 1/4; p = 1.0; 
    for (i = 0; i < 1000; i++) { 
      an = (a+b)/2;  
      bn = sqrt(a*b); 
      tn = t - p*(a-an)*(a-an);  
      p *= 2; 
      pi = (an+bn)*(an+bn)/(4*tn);  
      a = an; b = bn; t = tn; 
    }
    if ((pi - 3.14159265358979) <= 3.55271e-15) 
      return 1; //testas atliktas sekmingai
    return 0;  //teste ivyko klaida
  } 
\end{verbatim}
%}
Naudojant viengubo tikslumo slankaus kablelio kintamuosius buvo ie\v{s}komas 120 narys. Konstanta nuo pasirinktosios skiriasi $8.74228* 10^{-8}$
\begin{verbatim}
int floatTest() {
  volatile int i;
  volatile float an,bn,tn,pi;
  volatile float a,b,t,p;
  for(volatile int j = 0; j <9; j++) {
    a = 1.0; b = sqrt(0.5);
    t = 0.25; p = 1.0;
    for (i = 0; i < 120; i++) {
      an = (a+b)/2;
      bn = sqrt(a*b);
      tn = t - (p*(a-an)*(a-an));
      p *= 2;
      pi = (an+bn)*(an+bn)/(4*tn);
      a = an; b = bn; t = tn;
    } 
  }
  if ((pi - 3.14159265358979) <= 8.74228e-8)
    return 1; //testas atliktas sekmingai
  return 0; //teste ivyko klaida
}
\end{verbatim}
\subsubsection{Fiksuoto tikslumo algoritmas}
Fiksuoto tikslumo algoritmas:
\begin{equation}
b = \sum_{i=0}^{100000} (i * (-1)^{i+1})
\end{equation}

\begin{verbatim}
void intTest() {
  volatile int a,b;
  a = 1;
  b = 0;
  for(volatile int i = 0; i < 100000; i++) {
    b += i * a;
    a *= -1;
  }
}
\end{verbatim}
\subsubsection{Logini\k{u} funkcij\k{u} testavimo algoritmas}
Mikrovaldikliai atlieka logines AND, NOT, OR, XOR, bei bit\k{u} stumdymo funkcijas. 
\begin{verbatim}
short testBits() {
  volatile unsigned sum = 0x55555555; 
  // 0101 0101  0101 0101  0101 0101  0101 0101  
  for (volatile int i = 0; i < 0x020000; i++) {
    sum = sum << 1;          /// sum = 0xAAAAAAAA 
    sum &= 0x0000FFFF;       /// sum = 0x0000AAAA
    sum ^= 0xFFFFFFFF;       /// sum = 0xFFFF5555
    sum = sum << 16;         /// sum = 0x55550000
    sum = sum | (sum >> 16); /// sum = 0X55555555
    sum = ~sum;              /// sum = 0xAAAAAAAA
    sum = sum >> 1;          /// sum = 0x55555555
  }
  if (0x55555555 == sum)
    return 1; //testas ivykditas sekmingai
  return 0; // teste ivyko klaida
}
\end{verbatim}
\subsubsection{Greitas disckretinis Furj\.e\ eilut\.es transformavimo testas}
Naudotos ARM dsp\footnote{Dsp - digital signal processing $($skaitmeninis signal\k{u} apdorojimas$)$} bibliotekos, atskirai optimizuotos R4 bei M4 mikrovaldikliams. Naudota 1024 kompleksini\k{u} ta\v{s}k\k{u}, kurie masyve i\v{s}d\.estyti vienas po kito ir bendras masyvo ilgis 2048. Slankaus kablelio testui masyvas u\v{z}pildomas funkcija:
\begin{equation*}
Re(z) = (sin(v*i) + cos(v*7)),\; t = 0, 2, 4... 2046
\end{equation*}
\begin{equation}
Im(z) = 0
\end{equation}
\begin{verbatim}
#define TEST_LENGTH_SAMPLES 2048 
static float32_t testInput[TEST_LENGTH_SAMPLES]; 
static float32_t testOutput[TEST_LENGTH_SAMPLES/2]; 
uint32_t fftSize = 1024; 
uint32_t ifftFlag = 0; 
uint32_t doBitReverse = 1; 

void prepareTestFloat(void) {
  for(volatile int t = 0; t < 2048; t+=2) {
    testInput[i] = sin(t*3) + cos(t*7);
    testInput[i+1] = 0;
  }
}

int32_t testCfftFloat(void) { 
  arm_status status; 
  arm_cfft_radix4_instance_f32 S; 
  float32_t maxValue;  
   
  arm_cfft_radix4_init_f32(&S, fftSize, ifftFlag, doBitReverse); 
   
  arm_cfft_radix4_f32(&S, testInput); 
   
  arm_cmplx_mag_f32(testInput, testOutput, fftSize);  
   
  arm_max_f32(testOutput, fftSize, &maxValue, &testIndex); 
   
  if(testIndex == refIndex) { 
    return 1;
  }
  return 0;
} 
\end{verbatim}  
Fiksuoto tikslumo Furij\.e transformacijai naudoti ARM Q31 fiksuoto kablelio kintamieji. I\v{s} slankaus kablelio paversti \k{i} Q31 formato kintamuosius naudota funkcija$[$2$]$:
\begin{equation}
Q31(x)= ((int)\; ((x)*(float)(1<<31))\; )
\end{equation}   
Masyvo u\v{z}pildo funkcija:
\begin{equation*}
Re(z) = Q31(1000*(sin(v*i) + cos(v*7)))),\; t = 0, 2, 4... 2046
\end{equation*}
\begin{equation}
Im(z) = 0
\end{equation}
\begin{verbatim}
#define TEST_LENGTH_SAMPLES 2048
uint32_t fftSize = 1024; 
uint32_t ifftFlag = 0; 
uint32_t doBitReverse = 1; 
static q31_t testInputQ[TEST_LENGTH_SAMPLES]; 
static q31_t testOutputQ[TEST_LENGTH_SAMPLES/2]; 
uint32_t refQIndex = 0;

void prepareTestQ31(void) {
  double skaicius=0;
  int nulis = (int) ((0 * (float)(1<<31)));
  for(volatile int32_t i = 0; i < 2048; i+=2) {
    skaicius = 1000*sin(i*3) + 1000*cos(i*7);
    testInputQ[i] = ((int)(skaicius * (float)(1<<31)));
    testInputQ[i+1] = nulis;
  }
}

int32_t testCfftQ31(void) { 
  arm_status status; 
  arm_cfft_radix4_instance_q31 S; 
  q31_t maxValue; 
   
  status = arm_cfft_radix4_init_q31(&S, fftSize, ifftFlag, doBitReverse); 
   
  arm_cfft_radix4_q31(&S, testInputQ); 
  
  arm_cmplx_mag_q31(testInputQ, testOutputQ, fftSize);  
 
  arm_max_q31(testOutputQ, fftSize, &maxValue, &testIndex); 

  if(testIndex == refIndex) { 
    return 1;
  } 
  return 0;
}
\end{verbatim}
\subsection{\k{I}\.ejimo ir i\v{s}\.ejimo \k{i} tr\=uk\k{i} algoritmo apra\v{s}ymas}
Nor\.{e}ta palyginti laikus per kur\k{i} mikrovaldikliai gali duoti atsak\k{a} \k{i} i\v{s}orin\k{i} ar vidin\k{i} signal\k{a}. 
Naudojami maketai buvo paleisti veikti 100MHz grei\v{c}iu. Abu maketai keisdavo i\v{s}\.ejimo prievado vieno i\v{s} i\v{s}vad\k{u} login\k{i} lyg\k{i}, bei buvo \k{i}jungtas taimeris su tu\v{s}\v{c}ia paprograme. Osciloscopu matuotas i\v{s}vad\k{u} lygmen\k{u} kitimo laikas. Tik\.{e}tasi, kad Cortex-M4 mikrovaldiklis bus greitesnis.
\subsection{Srov\.es matavimai}
Nor\.{e}ta palyginti mikrovaldikli\k{u} naudojamas galias esant \k{i}vairiems darbo re\v{z}imams.
STM32F4DISCOVERY makete srov\.e matuota tiesiogiai. TMDXRM48USB maketas netur\.ejo 
trumpikli\k{u} tarp maitinimo ir mikrovaldiklio. Tarp maitinimo 3.3V i\v{s}\.ejimo ir mikrovaldiklio i\v{s}vad\k{u} maitinimo bei maitinimo 1.2V i\v{s}\.ejimo ir mikrovaldiklio logikos maitinimo buvo \k{i}terptos 0.12$\Omega$ var\v{z}os. 
\begin{figure}[H] %--< nustatymai kur bus paveiksliukas h - kaip tekste, t - puslapio virsuje, b - puslapio apacioje, H - tiksliai kaip tekste, p - atskiram paveiksleliu pslapy >--%
\centering %--< lygiavimas >--%
\includegraphics[scale=0.8]{pav/srovesc.jpeg} %--< scale - dydis, {aplankas/pav.pavadinimas} >--%
\captionsetup{labelformat=numbfirst} %--< pries tai apibrezem kaip turi atrodyti tekstas po paveiksleliu >--%
\captionsetup{labelseparator=tarpas}
\caption{Cortex-R4 srov\.{e}s matavimo schema. R=0.12 $\Omega$.}
\label{vienas}
\end{figure}
Vis\k{u} matavim\k{u} metu 
  i\v{s}vadai buvo nustatyti i\v{s}vesties re\v{z}ime ir 
  nustatyti nuliniame loginiame lygije. Buvo atlikti matavimai
  esant \v{s}ioms konfig\={u}racijoms:
\begin{itemize}
\item \k{i}jungtas stanby re\v{z}imas. Siekta i\v{s}matuoti minimal\k{u} srov\.{e}s sunaudojim\k{a} kai procesoriui nereikia atlikti darbo. 
\begin{itemize} 
\item Cortex-R4:
\begin{itemize}
\item Flash atmintis automatinio miego re\v{z}ime
\item procesoriai laukimo re\v{z}ime
\item taktuojama i\v{s} ma\v{z}os galios i\v{s}orinio kvarco 
\item periferijos i\v{s}jungtos
%damusti------------------------------------------%
\end{itemize}

\item Cortex-M4:
\begin{itemize}
\item vidinis 1.2V \k{i}tampos reguliatorius i\v{s}jungtas
\item Pll\footnote{PLL - phase lock loop da\v{z}ni\k{u} generatorius}, vidinis 16MHz ir i\v{s}orinis 8MHz taktiniai osciliatoriai atjungti. Veikia vidinis 32kHz taktinis osciliatorius
\item procesorius sustabdytas
\item mikrovaldiklis minimalioje srov\.es suvartojimo b\=usenoje.
\item re\v{z}im\k{a} galima i\v{s}eiti:
\begin{itemize} 
\item gavus i\v{s}orin\k{i} impuls\k{a} \k{i} NRST arba WKUP i\v{s}vadus 
\item realaus laiko laikrod\v{z}iui $($RTC$)$ sugeneravus tr\=uk\k{i}
\end{itemize} 
\end{itemize}
\end{itemize}
\item \k{i}jungtas negilaus miego re\v{z}imas esant \k{i}vairiems taktavimo grei\v{c}iams:
\begin{itemize} %aprasyti sleep modey
\item Cortex-R4:
\begin{itemize}
\item Flash atmintis automatinio miego re\v{z}ime
\item sistema veikia taktavimo da\v{z}niu
\item procesoriai laukimo re\v{z}ime
\item periferija atjungta
\end{itemize}
\item Cortex-M4:
\begin{itemize}
\item sistema veikia pll taktavimo da\v{z}niu
\item procesorius sustabdytas 
\item i\v{s} re\v{z}imo galima i\v{s}eiti betkokio tr\=ukio pagalba
\end{itemize}
\end{itemize}
\item esant skirtingiems taktavimo da\v{z}niams mikrovaldikliai atlikin\.ejo slankaus kablelio ir fiksuoto tikslumo aritmetikos testus, bei buvo paleisti veikti cikle: \begin{verbatim} while(1); \end{verbatim} 
\end{itemize}
\subsection{Radiacinis matavimas}
Siekiant patikrinti mikrovaldiklio veikim\k{a} ekstremaliam poveikiui. Buvo i\v{s}kelta hipotez\.{e}, kad ap\v{s}vie\v{c}iant jonizuojan\v{c}i\k{a}j\k{a} spinduliuote gali b\={u}ti sutrigdoma normali mikrovaldiklio veika:
\begin{itemize}
\item pa\v{z}eid\v{z}iant flash atmint\k{i} sutrikdoma programa, d\.{e}l ko mikrovaldiklis pradeda vykdyti neteisingas u\v{z}duotis arba j\k{u} i\v{s}vis nevykdyti
\item pa\v{z}eid\v{z}iant RAM atmint\k{i} gali b\={u}ti i\v{s}kraipomi skai\v{c}iavim\k{u} duomenys
\item sutrigdant \v{s}erdies atliekam\k{a} darb\k{a} gali b\={u}ti generuojamos skai\v{c}iavim\k{u}, logikos ir valdymo klaidos.
\end{itemize} 
Cortex-M4 ir Cortex-R4 mikrovaldikliai sujungti 4 laidu + \v{z}em\.{e} spi protokolu. Cortex-M4 nustatytas kaip master, Cortex-R4 kaip slave.
Cortex-M4 atlikin\.{e}jo matematinius skai\v{c}iavimus ir siunt\.{e} rezultatus \k{i} Cortex-R4 mikrovaldikl\k{i} patikrinimui, kuris per UART jungt\k{i} juos siunt\.{e} \k{i} kompiuterio terminal\k{a}.  
\begin{figure}[H] %--< nustatymai kur bus paveiksliukas h - kaip tekste, t - puslapio virsuje, b - puslapio apacioje, H - tiksliai kaip tekste, p - atskiram paveiksleliu pslapy >--%
\centering %--< lygiavimas >--%
\includegraphics[scale=0.6]{pav/raddiag.jpg} %--< scale - dydis, {aplankas/pav.pavadinimas} >--%
\captionsetup{labelformat=numbfirst} %--< pries tai apibrezem kaip turi atrodyti tekstas po paveiksleliu >--%
\captionsetup{labelseparator=tarpas}
\caption{Radiacinio matavimo logikos diagrama.}
\label{vienas}
\end{figure}
%------------------------------%
\newpage
   
\section{Rezultatai}
Cortex-R4 mikrovaldiklis beveik visus skai\v{c}iavimo testus atliko grei\v{c}iau nei cortex-M4 mikrovaldiklis esant 100MHz taktiniui da\v{z}niui. 
\begin{figure}[H] %--< nustatymai kur bus paveiksliukas h - kaip tekste, t - puslapio virsuje, b - puslapio apacioje, H - tiksliai kaip tekste, p - atskiram paveiksleliu pslapy >--%
\centering %--< lygiavimas >--%
\includegraphics[scale=0.8]{pav/testai.jpg} %--< scale - dydis, {aplankas/pav.pavadinimas} >--%
\captionsetup{labelformat=numbfirst} %--< pries tai apibrezem kaip turi atrodyti tekstas po paveiksleliu >--%
\captionsetup{labelseparator=tarpas}
\caption{Skai\v{c}iavimo algoritm\k{u} laikai. 1 - dvigubo tikslumo slankaus kablelio testas $($$($2.1.1$)$ algoritmas$)$. 2 - slankaus kablelio testas $($$($2.1.1$)$ antras algoritmas$)$. 3 - fiksuoto tikslumo kintam\k{u}j\k{u} testas $($$($2.1.2$)$ algoritmas$)$. 4 - logini\k{u} funkcij\k{u} testas $($$($2.1.3$)$ algoritmas$)$. 5 - Furij\.e slankaus kablelio $($$($2.1.4$)$ algoritmas$)$. 6 - Furij\.e fiksuoto tikslumo kintam\k{u}j\k{u} testas$($$($2.1.4$)$ antras algoritmas$)$.}
\label{vienas}
\end{figure}

Tirtas periferijos perjungimo greitis ir \k{i}\.ejimo i\v{s}\.ejimo i tu\v{s}\v{c}i\k{a} tr\=ukio paprogram\k{e} laikas esant 100MHz taktavimo da\v{z}niui.  
Cortex-R4 periferijos i\v{s}\.ejimo lygmen\k{i} keit\.{e} kas 280ns. \k{I}\.ejimo \k{i} tr\=uk\k{i} ir i\v{s}\.ejimo laikas 730ns. Skirtumo tarp IRQ ir FIQ tr\=uki\k{u} laiko nebuvo. Cortex-M4 periferikos i\v{s}vado login\k{i} lygmen\k{i} keit\.e kas 150ns. \k{I}\.ejimo \k{i} tr\=uk\k{i} ir i\v{s}\.ejimo laikas 200ns.
\begin{figure}[H] %--< nustatymai kur bus paveiksliukas h - kaip tekste, t - puslapio virsuje, b - puslapio apacioje, H - tiksliai kaip tekste, p - atskiram paveiksleliu pslapy >--%
\centering %--< lygiavimas >--%
\includegraphics[scale=0.4]{pav/trukiai.jpg} %--< scale - dydis, {aplankas/pav.pavadinimas} >--%
\captionsetup{labelformat=numbfirst} %--< pries tai apibrezem kaip turi atrodyti tekstas po paveiksleliu >--%
\captionsetup{labelseparator=tarpas}
\caption{Generuot\k{u} impuls\k{u} laikin\.e diagrama.}
\label{vienas}
\end{figure}
Mikrovaldiklio Cortex-R4 periferija tirta atskirai. Maitinama 3.3V.
\begin{itemize}
\item Prievadai nustatyti \v{z}emame loginiame lygyje:
\begin{itemize}
\item Naudajama galia 371.5mW
\end{itemize}
\item Periferija i\v{s}jungta:
\begin{itemize}
\item Periferijos i\v{s}vaduose atsiranda loginis auk\v{s}tas lygis:
\begin{itemize}
\item SPI4nCS, SPI4nENA
\item DMMnENA
\item CAN3TX, CAN3RX
\end{itemize}
\item Naudojama galia 341mW
\end{itemize}
\item Nurodomas ma\v{z}os galios osciliatorius periferij\k{u} taktavimui. Periferija i\v{s}jungiama
\begin{itemize}
\item I\v{s}jungiant periferij\k{a} tur\.{e}t\k{u} b\={u}ti automati\v{s}kai atjungiami ir osciliatoriai
\item Naudojama periferijos galia 283.25mW
\item $($Procesoriaus naudojama galia minimali (4mW)$)$
\end{itemize} 
\item I\v{s}jungiamos periferijos bei procesoriaus login\.{e} dalis atsakinga u\v{z} periferijos valdym\k{a}
\begin{itemize}
\item Mikrovaldiklis nustoja vykdyti kitas komandas
\item Naudojama periferijos galia 374mW
\item Padid\.{e}ja procesoriaus naudojama galia (16mW) lyginant su prie\v{s} tai atliktu matavimu.
\end{itemize}
\end{itemize}

Mikrovaldiklio Cortex-R4 \v{s}erdis maitinama 1.2V. Kadangi 3.3V maitinama ir kiti \k{i}renginiai esantys makete $($tokie kaip \v{s}viesos sesnosrius, akselerometras$)$, bei d\.{e}l galimos JTAG \k{i}takos rezultatams naudojama galia buvo skai\v{c}iuojama pagal formul\k{e} $P=I_{serdies} *1.2V$. Cortex-M4 maitinamas 3.3V. Skai\v{c}iuojant gali\k{a} daryta prielaida, kad i\v{s}jungus periferij\k{a} kit\k{u} \k{i}rengini\k{u} naudojama srov\.{e} yra apie 0mA. Cortex-M4 naudojama galia $P=I*3.3V$.
%--------------------%
\begin{figure}[H] %--< nustatymai kur bus paveiksliukas h - kaip tekste, t - puslapio virsuje, b - puslapio apacioje, H - tiksliai kaip tekste, p - atskiram paveiksleliu pslapy >--%
\centering %--< lygiavimas >--%
\includegraphics[scale=0.8]{pav/modes.jpg} %--< scale - dydis, {aplankas/pav.pavadinimas} >--%
\captionsetup{labelformat=numbfirst} %--< pries tai apibrezem kaip turi atrodyti tekstas po paveiksleliu >--%
\captionsetup{labelseparator=tarpas}
\caption{Mikrovaldikliai stanby re\v{z}ime. Cortex-R4 logikos naudojama galia ir Cortex-M4 naudojama galia.}
\label{vienas}
\end{figure}
\begin{figure}[H] %--< nustatymai kur bus paveiksliukas h - kaip tekste, t - puslapio virsuje, b - puslapio apacioje, H - tiksliai kaip tekste, p - atskiram paveiksleliu pslapy >--%
\centering %--< lygiavimas >--%
\includegraphics[scale=0.8]{pav/sleep.jpg} %--< scale - dydis, {aplankas/pav.pavadinimas} >--%
\captionsetup{labelformat=numbfirst} %--< pries tai apibrezem kaip turi atrodyti tekstas po paveiksleliu >--%
\captionsetup{labelseparator=tarpas}
\caption{Mikrovaldikli\k{u} galios vartojimo priklausomyb\.e nuo taktinio grei\v{c}io sleep re\v{z}ime.}
\label{vienas}
\end{figure}
\begin{figure}[H] %--< nustatymai kur bus paveiksliukas h - kaip tekste, t - puslapio virsuje, b - puslapio apacioje, H - tiksliai kaip tekste, p - atskiram paveiksleliu pslapy >--%
\centering %--< lygiavimas >--%
\includegraphics[scale=0.8]{pav/R4galia.jpg} %--< scale - dydis, {aplankas/pav.pavadinimas} >--%
\captionsetup{labelformat=numbfirst} %--< pries tai apibrezem kaip turi atrodyti tekstas po paveiksleliu >--%
\captionsetup{labelseparator=tarpas}
\caption{RM48 mikrovaldiklio naudojamos galios priklausomyb\.e nuo taktinio da\v{z}nio atliekant slankaus kablelio, fiksuoto tikslumo ir b\=unant tu\v{s}\v{c}iame cikle.}
\label{vienas}
\end{figure}
\begin{figure}[H] %--< nustatymai kur bus paveiksliukas h - kaip tekste, t - puslapio virsuje, b - puslapio apacioje, H - tiksliai kaip tekste, p - atskiram paveiksleliu pslapy >--%
\centering %--< lygiavimas >--%
\includegraphics[scale=0.8]{pav/M4galia.jpg} %--< scale - dydis, {aplankas/pav.pavadinimas} >--%
\captionsetup{labelformat=numbfirst} %--< pries tai apibrezem kaip turi atrodyti tekstas po paveiksleliu >--%
\captionsetup{labelseparator=tarpas}
\caption{STM32F4 mikrovaldiklio naudojamos galios priklausomyb\.e nuo taktinio da\v{z}nio atliekant slankaus kablelio, neslankaus kablelio ir b\=unant tu\v{s}\v{c}iame cikle.}
\label{vienas}
\end{figure}
Mikrovaldiklis Cortex-M4 buvo 1h \v{s}vitinamas 1.176MeV gama spinduliais $($137Cs$)$. Joki\k{u} nukripim\k{u} nuo normalaus mikrovaldiklio darbo neu\v{z}fiksuota.
Pereita prie \v{s}vitinimo rentgeno spinduliuote. Mikrovaldiklis buvo pad\.{e}tas po rentgeno lempa kurios \k{i}tampa bei srov\.{e} didinta laipsni\v{s}kai. Prie 19kV \k{i}tampa ir 19mA per 15min mikrovaldiklis nustojo veikti, ta\v{c}iau iki nustojimo veikti visus skai\v{c}iavumus atliko gerai. Mikrovaldiklis per 3h atsistat\.{e} ir v\.{e}l prad\.{e}jo atlikin\.{e}ti skai\v{c}iavimus.
\newpage
\section{I\v{s}vados}


\begin{enumerate}
\item Cortex-R4, esant 100MHz taktiniui da\v{z}niui, yra \v{z}ymiai greitesnis atliekant dvigubo tikslumo slankaus kablelio skai\v{c}iavimus nei cortex-M4 mikrovaldiklis.
%\item Du procesoriai esantys R4 stipriai padidina mikrovaldiklio naudojam\k{a} gali\k{a}
%\item RM48 mikrovaldiklio periferija dirba l\.{e}\v{c}iau nei stm32f4 esant 100MHz taktiniam da\v{z}niui.
\item Cortex-R4 tr\=uki\k{u} sistema lyginant su Cortex-M4 yra labai l\.{e}ta ir neefektyvi.
%\item Ma\v{z}iausia naudojama galia, kuri\k{a} pavyko pasiekti su RM48 yra 82 kartus didesn\.e nei ma\v{z}iausia vartojama galia, kuri\k{a} pavyko pasiekti su stm32f4 maketu.%
\item Mikrovaldikliams atliekant skirtingus skai\v{c}iavimus naudojama galia labiausiai priklauso nuo taktinio da\v{z}nio ir tik ne\v{z}ymiai nuo procesoriaus apkrovos skai\v{c}iavimais.
\item RM48 mikrovaldiklio periferija taisiklyngai nei\v{s}sijungia.
\item Su prieinamais jonizuojan\v{c}ios spinduliuot\.{e}s \v{s}altiniais atliekant matavimus net su vienguba \v{s}erdimi nepavyko pasiekti duomen\k{u} pa\v{z}eidim\k{u} programos vykdymo metu. Keliant intensyvum\k{a} prasid\.{e}jo i\v{s}liekantys procesoriaus funkcionalumo pa\v{z}eidimai.
\end{enumerate}

%Su prieinamai jonizuojančios spinduliuotės šaltiniais: ......... net su vienguba šerdimi nepavyko pasiekti duomenų pažeidimo programos vykdymo metu. Keliant intensyvumą pirma prasidėjo išliekantys procesoriaus funkcionalumo pažeidimai.

\newpage

%\section{Priedai}

%Bet kokia reikalinga papildoma informacija: paveiksl\.{e}liai, grafikai ir t.t.

%\end{onehalfspacing}

%\newpage

%\section*{Mokslin\.{e}s publikacijos ir konferencij\k{u} prane\v{s}imai}

%\addcontentsline{toc}{section}{Mokslin\.{e}s publikacijos ir konferencij\k{u} prane\v{s}imai}

%\begin{onehalfspacing}

%\v{S}iame skyriuje pateikiamos mokslin\.{e}s publikacijos bei \v{z}odiniai ir stendiniai prane\v{s}imai konferencijose, kurie yra susij\k{e} su "mano darbo pavadinimas".\\

%\begin{center}
%\textbf{Mokslini\k{u} publikacij\k{u}, \k{i}traukt\k{u} \k{i} mokslin\.{e}s informacijos instituto (ISI) pagrindini\k{u} \v{z}urnal\k{u} duomen\k{u} baz\k{e}, s\k{a}rašas\\}
%\end{center}

%1. 

%2. 

%\newpage

%\begin{center}
%\textbf{Kitos mokslin\.{e}s publikacijos\\}
%\end{center}

%1. 

%2. 

%\begin{center}
%\textbf{\v{Z}odiniai prane\v{s}imai konferencijose}\\
%\end{center}

%1. 

%2. 

%\begin{center}
%\textbf{Stendiniai prane\v{s}imai konferencijose}\\
%\end{center}

%1. 

%2. 


%\end{onehalfspacing}

%\newpage

%%-----====literaturos sarasas====------%%
\section*{Literat\={u}ros s\k{a}ra\v{s}as}
\addcontentsline{toc}{section}{Literat\={u}ros s\k{a}ra\v{s}as}
\begin{enumerate}
\item $\pi$ skai\v{c}iavimo algoritmas. \begin{verbatim}http://en.wikipedia.org/wiki/Gauss-Legendre_algorithm\end{verbatim}
\item  Q formatai ir konvertavimas. \begin{verbatim}http://infocenter.arm.com/help/topic/com.arm.doc.dai0033a/
DAI0033A_fixedpoint_appsnote.pdf\end{verbatim}
\item TMDXRM48USB trumpas apra\v{s}as. \begin{verbatim}http://processors.wiki.ti.com/images/5/5c/RM48_USB_QUICK_START.pdf\end{verbatim}
\item Detalus Cortex-R4 apra\v{s}as. \begin{verbatim}http://www.ti.com/lit/ds/spns177a/spns177a.pdf\end{verbatim}
\item Cortex-R4 energijos vartojimo optimizavimo apra\v{s}as.\begin{verbatim} http://www.ti.com/lit/an/spna172a/spna172a.pdf\end{verbatim}
\item ARM architekt\={u}r\k{u} apra\v{s}ai. \begin{verbatim}http://www.arm.com/products/index.php\end{verbatim}
\item Cortex-R4 silicon errata. \begin{verbatim}
http://www.ti.com/lit/er/spnz194b/spnz194b.pdf \end{verbatim}
\item STM32F4-Discovery trumpas apra\v{s}as. \begin{verbatim}http://www.st.com/internet/com/TECHNICAL_RESOURCES/
TECHNICAL_LITERATURE/DATA_BRIEF/DM00037955.pdf\end{verbatim}
\item Detalus Cortex-M4 mikrovaldiklio apra\v{s}as. \begin{verbatim}http://www.st.com/internet/com/TECHNICAL_RESOURCES/TECHNICAL_LITERATURE/
DATASHEET/DM00037051.pdf\end{verbatim}
\item Cortex-M4 slankaus kablelio modulio sovybi\k{u} apra\v{s}as. \begin{verbatim}http://www.st.com/internet/com/TECHNICAL_RESOURCES/TECHNICAL_LITERATURE/
DATASHEET/DM00037051.pdf\end{verbatim}
\end{enumerate}

%\bibliography{Kursinis.bbl} %--< skliausteliuose turi buti nurodamas literaturos saraso failo pavadinimas >--%
%\bibliographystyle{nature} %--< literaturos stilius >--%

\newpage

\section*{Santrauka}


\addcontentsline{toc}{section}{Santrauka}

%\begin{onehalfspacing}

\begin{center}

%\studentas \\[12pt]

\MakeUppercase{,,\pavadinimas''}\\

\end{center}

U\v{z}duotys kurioms reikia didelio patikimumo reikalauja skai\v{c}iavim\k{u} patikrinimo. Cortex-R4 turi integruot\k{a} skai\v{c}iavimo ir kontrol\.es dubliavim\k{a}, ta\v{c}iau neai\v{s}ku kaip tai yra energeti\v{s}kai naudinga. \\
\indent
\v{S}iame darbe buvo pabandyta i\v{s}siai\v{s}kinti Hercules \v{s}eimos atstovo RM48 mikrovaldiklio energetin\k{i} efektyvum\k{a}, matematini\k{u} veiksm\k{u} su \k{i}vairiais kintamaisiais algoritm\k{u} atlikimo spart\k{a}, \k{i}\.{e}jimo ir i\v{s}\.{e}jimo \k{i} tr\={u}ki laikus ir palyginant su kitu populiariu mikrovaldikliu. Bandyta patikrinti ir mikrovaldikli\k{u} veikimo ypatybes veikiant jonizuojan\v{c}i\k{a}ja spinduliuote. \\
\indent
Rezultatai rodo, kad Cortex-R4 mikrovaldiklis matematinius algoritmus atlieka grei\v{c}iau nei Cortex-M4 mikrovaldiklis, ta\v{c}iau jam nusileid\v{z}ia tr\={u}ki\k{u} sistemos ir energetiniu efektyvumu. Radiaciniai matavimai rodo, kad mikrovaldikliai gerai veikia tol, kol atsiranda i\v{s}liekantys funkcionalumo pa\v{z}eidimai.  



%\newpage

\section*{Summary}

\addcontentsline{toc}{section}{Summary}

\begin{center}

%\studentas \\[12pt]

\MakeUppercase{,,Analysis of Cortex-R4 Architecture Microcontroller Properties''}\\


\end{center}

Tasks that require high reliability need computation verification. Cortex-R4 has an integrated computing mirroring of calculation and control, but it is unclear how energeticaly efficent it is. \\
\indent In this paper, an attempt was made to find out energy efficiency, speed of execution of mathematical algorithms with different variables, time of getting in and out of an interrupt of Hercules family representative xRM48 microcontroller and to compare it with another popular microcontroller. Also an attempt was made to check operation properties of microcontroller under ionizing radiation. \\
\indent
The rezults show that the Cortex-R4 microcontroller performs faster computations, but fades in comparison to Cortex-M4 in interrupt times and energy efficiency. Raidation mesurements show that the microcontrollers work well until persistent functional damage stops their work.
%Text

\end{onehalfspacing}


\end{document} 