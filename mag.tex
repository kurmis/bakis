\documentclass[a4paper, 12pt]{article} %--< dokumento nustatymai >--%
%\usepackage[T1]{fontenc} %--< reikia kuomet norima naudoti komanda \k{a} - a raide bus parasyta su nosine >--%
\usepackage{graphicx} %--< reikalinga paveiksliuku ikelimui >--%
\usepackage[left=1.18in,top=0.79in,right=0.39in,bottom=0.79in]{geometry} %--< parasciu nustatymai >--%
\usepackage{color} %--< spalvoto srifto nustatymai >--%
\usepackage{setspace} %--< setspace.sty failas turi buti toje pacioje direktorijoje kaip ir winedt failas >--%
\usepackage{caption} %--< leidzia rasyti komentarus po objektais >--%
\usepackage{cite} %-< reikalinga citavimui >--%
\usepackage{verbatim} %--< leidzia naudoti \begin{comment}...\end{comment} funkcijas>--%

\usepackage[utf8]{inputenc} %--< kad butu galima rasyti lietuviskai >--%
\usepackage[L7x]{fontenc}
\usepackage[lithuanian]{babel}
%\usepackage{lmodern}

%\usepackage[english]{babel}
\usepackage{array}
\usepackage[titles]{tocloft}
\usepackage{amsmath} %--< formuliu rasymui >--%
\usepackage{indentfirst} %--< pirmas paragrafas prasideda su tab'u >--%
\usepackage{enumitem} %--< tarpu tarp eiluciu sarase redagavimui >--%
%\usepackage{numline} %--< eiluciu numeravimas, sio paketo naudojimas panaikina section pavadinimus >--%
\usepackage{xfrac}
\usepackage{float}
\usepackage{array}
\usepackage{multirow}
\usepackage[]{SIunits}


\makeatletter
\let\c@lofdepth\relax
\let\c@lotdepth\relax
\makeatother
\usepackage{subfigure}

\setlength{\cftaftertoctitleskip}{-10pt}
\setlength{\cftbeforetoctitleskip}{-10pt}
%\setlength{\parskip}{6pt} %--< kiek tekstas nutoles nuo paragrafo pavadinimo >--%
\parindent=36pt %--< tab'o dydis 1 inch = 72 pt >--%

%-----------< kaip atrodo paveiksleliu ir lenteliu tesktas >-----------------%
\DeclareCaptionLabelFormat{numbfirst}{#2\bothIfSecond{\nobreakspace}{pav}} % -< paveiksliukai vadinami "pav" >- %
\DeclareCaptionLabelFormat{numbtwo}{#2\bothIfSecond{\nobreakspace}{lentel\.{e}}} % -< lenteles vadinamos "lentele" >- %
\DeclareCaptionLabelSeparator{tarpas}{. } % -< po pav taskas ir tarpas >- %
\captionsetup{justification=centering, font=normalsize, labelfont={}, textfont={it}}



%-----------------------------< keletas naudingu komandu >------------------------------------------%
\begin{comment}

-----< Lietuviskos raides, jei neveikia rasymas lietuviskai >--------
\k{a} = ą
\v{z} = ž
\v{s} = š
\.{e} = ė 
\k{e} = ę
\={u} = ū
\k{u} = ų
\k{i} = į
\v{c} = č
----------------------------------------

\textbf{padaro teksta bold}
\textit{Padaro teksta italic}
\cite{ka cituojam}
\newpage naujas puslapis

Jei norim tekste formuliu rasom ja tarp $formule$, jei numeruotai, tai pagal
\begin{equation}
formule
label{formules_label}
\end{equation}
ir naudojam formules citavimui \ref{formule_label}

\noindent pritaukia eilute prie krasto

-------< paveiksliukams patogu susikurti atskira folderi takim paveiksliukai ar pav >--------
-------< vienas paveiksliukas >---------
\begin{figure}[H] %--< nustatymai kur bus paveiksliukas h - kaip tekste, t - puslapio virsuje, b - puslapio apacioje, H - tiksliai kaip tekste, p - atskiram paveiksleliu pslapy >--%
\centering %--< lygiavimas >--%
\includegraphics[scale=0.44]{pav/SPP_dispersijos_kreive} %--< scale - dydis, {aplankas/pav.pavadinimas} >--%
\captionsetup{labelformat=numbfirst} %--< pries tai apibrezem kaip turi atrodyti tekstas po paveiksleliu >--%
 \captionsetup{labelseparator=tarpas}
 \caption{Tesktas po paveiksleliu}
 \label{vienas}
 \end{figure}

-------< du paveiksliukai ir t.t. >---------
\begin{figure}[h]
\centering
\subfigure[]{ %--< pirmas paveiksliukas, jei [] tusti tai numeruojama raidemis >--%
\includegraphics[scale=0.3]{pav/pirmas}
\label{pirmas}
}
\centering
\subfigure[]{ %--< antras paveiksliukas >--%
\includegraphics[scale=0.27]{antras}
 \label{antras}
 }
\captionsetup{labelformat=numbfirst}
\captionsetup{labelseparator=tarpas}
\caption{Kas paveikslikuose (a) ir (b), tesktas po paveiksliuku}
\label{dupaveiksliukai}
\end{figure}

----------< numeruojam teksta >-------
\begin{enumerate}[itemsep=12pt, parsep=-10pt]
 \item[a)]  %--< jei [] nieko arba be [], tai numeruos skaiciais
\item[b)] 
\item[c)]      
\end{enumerate}

\end{comment}

%=================================================================%
%--< dalis kuria reikia uzpildyti, bet tam tikrais atvejais gali tekti pagal poreikius paredaguoti title.tex faila >--%
%=================================================================%

\newcommand{\studentas}{Mindaugas Kurmauskas} %--< cia irasyti savo varda pavarde (toliau kitus duomenis) ir nebereikes tekste >--%
\newcommand{\vadovas}{a. Mindaugas Eglinskas}
\newcommand{\recenzentas}{...}
\newcommand{\vedejas}{prof. Ved\.{e}jas}
\newcommand{\pavadinimas}{Ketursraig\v{c}io sraigtasparnio lokalizacija} %--< cia irasyti darbo pavadinima >--%
\newcommand{\engpavadinimas}{Quadcopter localization}

\newcommand{\katedra}{informatikos katedra} %--< irasyti savo, nebutinai komanda >--%
%=======%
\newcommand{\kkek}{Kieto k\={u}no elektronikos katedra}
\newcommand{\kek}{Kvantin\.{e}s elektronikos katedra}
\newcommand{\pfk}{Puslaidininki\k{u} fizikos katedra}
\newcommand{\rfk}{Radiofizikos katedra}
\newcommand{\tfk}{Teorin\.{e}s fizikos katedra}
\newcommand{\bfsk}{Bendrosios fizikos ir spektroskopijos katedra}
%==========%

\newcommand{\darbas}{Pasiruo\v{s}imo magistriniam darb\k{u} ataskaita} %--< irasyti savo >--%
%=========%
\newcommand{\bakikurs}{Pagrindini\={u} studij\={u} kursinis darbas}
\newcommand{\bakis}{Pagrindini\={u} studij\={u} baigiamasis darbas}
\newcommand{\magikurs}{Magistrant\={u}ros studij\k{u} kursinis darbas}
\newcommand{\magis}{Magistrant\={u}ros studij\k{u} baigiamasis darbas}
%==========%

\newcommand{\studprog}{Informatika} %--< irasyti savo, nebutinai komanda >--%
%==========%
%--< besimokantiems fizfake studiju programos >--%
\newcommand{\fiz}{Fizika}
\newcommand{\tf}{Taikomoji fizika}
\newcommand{\tfe}{Telekomunikacij\={u} fizika ir elektronika}
\newcommand{\kf}{Kompiuterin\.{e} fizika}
\newcommand{\mtfv}{Moderni\k{u}j\k{u} technologij\k{u} fizika ir vadyba}
\newcommand{\bef}{Branduolin\.{e}s energetikos fizika}
\newcommand{\apchef}{Aplinkos ir chemin\.{e} fizika}
\newcommand{\bio}{Biofizika}
\newcommand{\laztech}{Lazerin\.{e}s technologijos}
\newcommand{\fiztech}{Fizikin\.{e}s technologijos ir j\k{u} vadyba}
\newcommand{\lazfiz}{Lazerin\.{e} fizika ir optin\.{e}s technologijos}
\newcommand{\mpf}{Med\v{z}iagotyra ir puslaidininki\k{u} fizika}
\newcommand{\optel}{Optoelektronikos med\v{z}iagos ir technologijos}
\newcommand{\tfem}{Telekomunikacij\k{u} fizika ir elektronika}
\newcommand{\tfa}{Teorin\.{e} fizika ir astrofizika}
%=================================================================%
%=================================================================%


%--------------------------------------< nuo cia prasideda dokumento tekstas >-------------------------------------%


\begin{document} % visas tekstas turi buti tarp begin{document} ir end{document} komandu


\begin{titlepage}

\begin{center}


% Upper part of the page

\MakeUppercase{\normalsize \textbf{Vilniaus universitetas}}\\
\MakeUppercase{\normalsize \textbf{Matematikos ir informatikos fakultetas}}\\
\MakeUppercase{\normalsize \textbf{\katedra}}\\[180pt]
%\normalsize \studentas \\ [24pt]

% Title

\MakeUppercase{{ \normalsize \pavadinimas}}\\[24pt]
\MakeUppercase{{ \normalsize \engpavadinimas}}\\[24pt]

\normalsize \darbas \\ [24pt]
(studij\k{u} programa -- \MakeUppercase{\normalsize \studprog)}\\ [130pt]

\linespread{1.3}
% Author and supervisor
\begin{minipage}{0.7\textwidth}
\begin{flushleft} \normalsize

Studentas \\  %--< nekeiciam >--%
Darbo vadovas \\ %--< nekeiciam >--%
Recenzentas\\ %--< nekeiciam >--%
\end{flushleft}
\end{minipage}
\begin{minipage}{0.29\textwidth}
\begin{flushleft} \normalsize
\studentas \\
\vadovas \\
\recenzentas \\
\end{flushleft}
\end{minipage}

\vfill

% Bottom of the page
{\normalsize Vilnius \the\year}

\end{center}

\end{titlepage}  % itraukiam titulini lapa i visa dokumenta

\newpage

\pagenumbering{arabic}
\setcounter{page}{1} % nuo kurio skaitmens pradeda numeruoti puslapius

%\tocloftpagestyle{abbrv}
%\renewcommand{\cftsecleader}{\cftdotfill{\cftdotsep}} % turinio isvaizdos koregavimas

%\renewcommand{\multirowsetup}{\centering}

%\renewcommand{\contentsname}{Turinys}
%\renewcommand\refname{Literat\={u}ros s\k{a}ra\v{s}as} % pakeicia "references" i "literaturos sarasas"
%\tableofcontents

%\newpage

\begin{onehalfspacing}

\section*{Ap\v{z}valga bei tyrimo problemos apra\v{s}ymas}

\section{Tyrimo objektas}

Tyrimo objektas yra ketursraigt\v{c}io sraigtasparnio lokalizacijos ir \v{z}em\.elapio sudarymas, esant ribotiems skai\v{c}iavimo resursams.
\\
\indent Paprastas ketursraigtis sraigtasparnis n\.era paj\.egus su savimi skraidinti modernaus kompiuterio atlikti sud\.etingiems skai\v{c}iavimams. Skai\v{c}iavimams supaprastinti taipat galima naudoti ir paprastenius sensorius tokius kaip lazerin\k{i} atstumo matuokl\k{i}, kuris pasi\v{z}ymi dideliu tikslumu, tav\v{c}iau yra gan\.{e}tinai brangus. Kaip u\v{z}tikrinti  realaus laiko sraigtasparnio lokalizacij\k{a} ir \v{z}em\.elapio sudarym\k{a} esant ribotiems skai\v{c}iavimo resursams naudojant paprastus sensorius ir yra \v{s}io tyrimo objektas. 

\section{Darbo tikslai ir u\v{z}daviniai}
Darbo tikslas yra sukurti ketursraigt\k{i} sraigtasparn\k{i} ir naudojantis jame esan\v{c}iu kompiuteriu atlikti realaus laiko sinchronin\k{i} lokalizavim\k{a} ir \v{z}em\.elapio sudarym\k{a}.
\\
\indent Siekiant \v{s}io tikslo turi b\=uti i\v{s}pr\k{e}sti \v{s}ie u\v{z}davinai:
\begin{itemize}
\item sukonstuotas ketursraigtis sraigtasparnis
\item sukurta jo skraidymo ir stabilizacijos sistema
\item pritaikytas FastSLAM algoritmas\cite{Zikos2011}
\item parinkti parametrai \v{z}em\.elapio detalumo sudarymui
\end{itemize}
     
  
\section{Tyrimo aktualumas}
Ketursraig\v{c}iai sraigtasparniai pasi\v{z}ymi stabilumu, bei d\.el ma\v{z}ai
 judan\v{c}i\k{u} daliu patvarumu ir paprastumu. Jie gali patekti \k{i} \v{z}mon\.ems
  sunkiai prieinamas vietas bei gali skraidinti nedidelius krovinius tokius kaip pirmosios
   pagalbos priemones, tod\.el jie turi didel\k{i} potencial\k{a} pasitarnauti nutikus 
   nelaimei. Dabar paprastai jie yra valdomi nuotiliniu b\=udu ant \v{z}em\.es esan\v{c}io
    piloto. Pilotus ruo\v{s}ti yra sud\.etinga, bei vienas pilotas gali valdyti tik 
    vien\k{a} sraigtasparn\k{i} vienu metu, tod\.el yra verta kurti 
    autonominius sraigtasparnius.
\\
\indent Pramon\.eje yra svarbu ne tik patikimumas, bet ir pagaminimo ka\v{s}tai. Dabar naudojami sprendimai susideda i\v{s} brangi\k{u} akumuliatori\k{u}, galing\k{u} mikro kompiuteri\k{u}, stereoskopini\k{u} kamer\k{u} ir lazerini\k{u} atstumo matuokli\k{u}\cite{Nguyen2007}. Paprastai tokia \k{i}ranga b\=una karinio lygio, o biud\v{z}etiniai variantai konstruojami universitetuose ant va\v{z}in\.ejan\v{c}i\k{u} platform\k{u}\cite{Vincke2010a}\cite{Longchamp2010}, o skrendantys aparatai paprastai naudoja ant \v{z}em\.es sumontuotas duomen\k{u} apdorojimo stoteles bei infraraudonuosius sensorius sraigtasparnio pozicijai nustatyti. 

\section{Tyrimo metodika}

Darbo eigoje bus parinkti konkret\=us sensoriai, bei \k{i}vertinus numatom\k{a} algoritmo sudetingum\k{a} parinktas mikro valdiklis. Skryd\v{z}io metu bus fiksuojama mikro valdiklio apkrova skai\v{c}iavimais bei naudojama energija. Atsi\v{z}velgiant \k{i} \v{s}iuos duomenis bus koreguojami \v{z}em\.elapio sudarymo detalumas ir sensori\k{u} nuskaitymo da\v{z}nis. Atlikus korekcijas bus tikrinama sraigtasparnio lokalizacijos tikslumas.

\section{Laukiami rezultatai}

Magistrinio darbo metu planuojama patobultinti metod\k{a} sraigtasparnio autonominei trimatei sinchroninei lokalizacijai ir \v{z}em\.elapio sudarymui. 


\end{onehalfspacing}

%\newpage



%%-----====literaturos sarasas====------%%

\addcontentsline{toc}{section}{Literat\={u}ros s\k{a}ra\v{s}as}
\bibliography{library} %--< skliausteliuose turi buti nurodamas literaturos saraso failo pavadinimas >--%
\bibliographystyle{alpha} %--< literaturos stilius >--%

%\newpage

%\section*{Santrauka}


%\addcontentsline{toc}{section}{Santrauka}

%\begin{onehalfspacing}

%\begin{center}

%\studentas \\[12pt]

%\MakeUppercase{,,\pavadinimas''}\\

%\end{center}

%Tekstas

%\newpage

%\section*{Summary}

%\addcontentsline{toc}{section}{Summary}

%\begin{center}

%\studentas \\[12pt]

%\MakeUppercase{,,Title''}\\


%\end{center}

%Text

%\end{onehalfspacing}

\end{document} 