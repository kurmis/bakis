\documentclass[12pt, a4paper, lithuanian]{article}

\usepackage[utf8x]{inputenc}
\def\LTfontencoding{L7x}
\usepackage[\LTfontencoding]{fontenc}
\usepackage[lithuanian]{babel}

\usepackage{VUMIF}
% \usepackage[mathcsdepttitle]{VUMIF} % --- matematinÄ—s informatikos katedros
%     titulinio puslapio formatavimas

% Titulinio puslapio reikalai
\vumifdept{Informatikos katedra}
\title{Ketursraig\v{c}io sraigtasparnio lokalizacija\\Quadcopter localization}
\vumifpaper{Pasiruo\v{s}imo magistro baigiamajam darbui ataskaita}

\author{
    Mindaugas Kurmauskas
}

\supervisor{a. Mindaugas Eglinskas}
\reviewer{lektorius Irus Grinis}
\date{Vilnius \\ 2013}

%--------------------------------------< nuo cia prasideda dokumento tekstas >-------------------------------------%


\begin{document} % visas tekstas turi buti tarp begin{document} ir end{document} komandu


\maketitle
\newpage

\pagenumbering{arabic}
\setcounter{page}{1} % nuo kurio skaitmens pradeda numeruoti puslapius

%\tocloftpagestyle{abbrv}
%\renewcommand{\cftsecleader}{\cftdotfill{\cftdotsep}} % turinio isvaizdos koregavimas

%\renewcommand{\multirowsetup}{\centering}

%\renewcommand{\contentsname}{Turinys}
%\renewcommand\refname{Literat\={u}ros s\k{a}ra\v{s}as} % pakeicia "references" i "literaturos sarasas"
%\tableofcontents

%\newpage

\begin{onehalfspacing}

\section*{Ap\v{z}valga bei tyrimo problemos apra\v{s}ymas}

\section{Tyrimo objektas}

Tyrimo objektas yra ketursraigt\v{c}io sraigtasparnio lokalizacijos ir \v{z}em\.elapio sudarymas, esant ribotiems skai\v{c}iavimo resursams.
\\
\indent Paprastas ketursraigtis sraigtasparnis n\.era paj\.egus su savimi skraidinti modernaus kompiuterio atlikti dabar naudojamiems lokalizacijos ir \v{z}em\.elapio sudarymo algoritm\k{u} skai\v{c}iavimams realiu laiku. Skai\v{c}iavimams supaprastinti galima naudoti paprastenius sensorius tokius kaip lazerin\k{i} atstumo matuokl\k{i}, kuris pasi\v{z}ymi dideliu tikslumu, ta\v{c}iau yra gan\.{e}tinai brangus ir turi siaur\k{a} matavimo spindul\k{i}. Kaip u\v{z}tikrinti  realaus laiko sraigtasparnio lokalizacij\k{a} ir \v{z}em\.elapio sudarym\k{a} esant ribotiems skai\v{c}iavimo resursams naudojantis paprastais jutikliais ir yra \v{s}io tyrimo objektas. 

\section{Darbo tikslai ir u\v{z}daviniai}
Darbo tikslas yra sukurti ketursraigt\k{i} sraigtasparn\k{i} ir naudojantis jame esan\v{c}iu kompiuteriu atlikti realaus laiko sinchronin\k{i} trimat\k{i} lokalizavim\k{a} ir \v{z}em\.elapio sudarym\k{a}. 
\\
\indent Siekiant \v{s}io tikslo turi b\=uti i\v{s}pr\k{e}sti \v{s}ie u\v{z}davinai:
\begin{itemize}
\item Sukonstruotas ketursraigtis sraigtasparnis.  
\item Sukurta jo skraidymo ir stabilizavimo sistema. Kadangi sraigtasparnis kuriamas i\v{s} atskir\k{u} dali\k{u} matematinius algoritmus reikia pritaikyti esamai sistemai.
\item Pritaikytas FastSLAM algoritmas trima\v{c}iui lokalizavimui\cite{Zikos2011}. Did\.ejant surinkt\k{u} duomen\k{u} erdvei ma\v{z}iausiai auga skai\v{c}iavim\k{u} kiekis.
\item Parinkti parametrai \v{z}em\.elapio detalumo sudarymui.
\end{itemize}
     
  
\section{Tyrimo aktualumas}
Ketursraig\v{c}iai sraigtasparniai pasi\v{z}ymi stabilumu, bei d\.el ma\v{z}ai
 judan\v{c}i\k{u} dali\k{u} patvarumu ir paprastumu. Jie gali patekti \k{i} \v{z}mon\.ems
  sunkiai prieinamas vietas bei gali skraidinti nedidelius krovinius tokius kaip pirmosios
   pagalbos priemones, tod\.el jie turi didel\k{i} potencial\k{a} pasitarnauti nutikus 
   nelaimei. Dabar paprastai jie yra valdomi nuotiliniu b\=udu ant \v{z}em\.es esan\v{c}io
    piloto. Pilotus ruo\v{s}ti yra sud\.etinga, bei vienas pilotas gali valdyti tik 
    vien\k{a} sraigtasparn\k{i} vienu metu, tod\.el yra verta kurti 
    autonominius sraigtasparnius.\\
    \indent Neseniai kompanijos Amazon ir UPS paskelb\.e, kad naudos sraigtasparnius krovini\k{u} gabenimui. Pramon\.eje yra svarbu ne tik patikimumas, bet ir pagaminimo ka\v{s}tai. Dabar naudojami sprendimai susideda i\v{s} brangi\k{u} akumuliatori\k{u}, galing\k{u} kompiuteri\k{u}, stereoskopini\k{u} kamer\k{u} ir lazerini\k{u} atstumo matuokli\k{u}\cite{Nguyen2007}. Kolkas biud\v{z}etiniai variantai konstruojami universitetuose ant va\v{z}in\.ejan\v{c}i\k{u} platform\k{u}\cite{Vincke2010a}\cite{Longchamp2010}, o skrendantys aparatai paprastai naudoja ant \v{z}em\.es sumontuotas duomen\k{u} apdorojimo stoteles bei i\v{s}orinius infraraudonuosius sensorius sraigtasparnio lokalizacijai.
    \\
    \indent Pasirinktas FastSLAM algoritmas pagal skaitytus straipsnius labiausiai atitinka norimus pasiekti tikslus. Lyginant su prapl\.estu Kalmano filtro (EKF), suspaustu prapl\.estu Kalmano filtru (CEKF) bei bekvapiu Kalmano filtru (unscented Kalman filter UKF)\cite{6389151} SLAM algoritmais, FastSLAM n\.era toks tikslus, ta\v{c}iau jam atlikti u\v{z}tenka ma\v{z}iau skai\v{c}iavim\k{u}\cite{Zikos2011}. 


\section{Tyrimo metodika}
Darbo eigoje bus parinkti konkret\=us jutikliai, bei \k{i}vertinus numatom\k{a} algoritmo sudetingum\k{a} parinktas mikrovaldiklis. Skryd\v{z}io metu bus fiksuojama mikrovaldiklio apkrova skai\v{c}iavimais bei naudojama energija. Atsi\v{z}velgiant \k{i} \v{s}iuos duomenis bus koreguojami \v{z}em\.elapio sudarymo detalumas ir jutikli\k{u} nuskaitymo da\v{z}nis. Atlikus korekcijas bus tikrinama sraigtasparnio lokalizacijos tikslumas.
\\
\indent Dabar valdymui bei duomen\k{u} surinkimui naudojamas STM32F4-Discovery maketas, kuriame yra STM32F303 mikrovaldiklis, bei  L3FD20 giroskopas ir LSM303DLHC akselerometras bei magnetrometras. Programavimas atliekamas KEIL programa, c ir c++ kalbomis. 
\\
Galimi mikrovaldikliai ir jutikliai:
\begin{itemize}
\item BeagleBoard-xM - AM37x 1GHz \v{s}erdis.
\item fit-PC2 - Intel Atom Z5xx 1.1-1.6GHz \v{s}erdis.
\item 2 x TCM8230MD - 640x480 kamera.
\item CMUcam4 - kamera su integruotu mikrovaldikliu vaizdo apdorojimui.
\item KS101B ultragarsinis atstumo matuoklis. Matuojamas atstumas iki 6m. Auk\v{s}\v{c}iui vir\v{s} \v{z}em\.es matuoti.


\end{itemize} 

\section{Laukiami rezultatai}

Magistrinio darbo metu planuojama patobultinti metod\k{a} sraigtasparnio autonominei trimatei sinchroninei lokalizacijai ir \v{z}em\.elapio sudarymui. 


\end{onehalfspacing}





%%-----====literaturos sarasas====------%%

\addcontentsline{toc}{section}{Literat\={u}ros s\k{a}ra\v{s}as}
\bibliography{library} %--< skliausteliuose turi buti nurodamas literaturos saraso failo pavadinimas >--%
\bibliographystyle{alpha} %--< literaturos stilius >--%

%\newpage

%\section*{Santrauka}


%\addcontentsline{toc}{section}{Santrauka}

%\begin{onehalfspacing}

%\begin{center}

%\studentas \\[12pt]

%\MakeUppercase{,,\pavadinimas''}\\

%\end{center}

%Tekstas

%\newpage

%\section*{Summary}

%\addcontentsline{toc}{section}{Summary}

%\begin{center}

%\studentas \\[12pt]

%\MakeUppercase{,,Title''}\\


%\end{center}

%Text

%\end{onehalfspacing}

\end{document} 